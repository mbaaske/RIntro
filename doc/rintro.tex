% 
\documentclass[10pt]{beamer} 
\mode<presentation>{%
	\setbeamertemplate{section in toc}[sections numbered]%
	\setbeamertemplate{subsection in toc}[subsections numbered] 
}
\usepackage[T1]{fontenc} 	  % Fontencoding

\usepackage{Sweave}



\usepackage[utf8]{inputenc}

% R -----------------------------------------------
\let\proglang=\textsf
\newcommand{\pkg}[1]{{\fontseries{b}\selectfont #1}}

\title{\proglang{R} Basics and Examples - A short introduction}
\author{Markus Baaske}

\begin{document}

\maketitle

\begin{frame}{The R Project for Statistical Computing}
The \proglang{R} project \url{http://www.r-project.org} develops a free
software environment for statistical computing and graphics. \proglang{R}
compiles and runs on a wide variety of UNIX platforms, Windows and MacOS, is mostly used for statistics but can also be
used as a programming (script) language alone. \par
\proglang{R} is organized as a core distribution of base packages which can be
extended by further packages loaded into the a user workspace (or interpreter
global environment).\par
Some useful links are
\begin{itemize}
  \item Tutorials on using \proglang{R} can be
  found at \url{http://www.r-tutor.com/}
  \item Meta search and package
  documentation \url{https://www.rdocumentation.org/}
  \item \proglang{R} CRAN repository for contributed packages:
  \url{https://cran.r-project.org/}
  \item A short reference card \url{https://cran.r-project.org/doc/contrib/Short-refcard.pdf}
\end{itemize}
\end{frame}

\begin{frame}[fragile]{\proglang{R} Basics}
\end{frame}

\end{document}
