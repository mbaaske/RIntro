% 
\documentclass[10pt]{beamer} 
\mode<presentation>{%
	\setbeamertemplate{section in toc}[sections numbered]%
	\setbeamertemplate{subsection in toc}[subsections numbered] 
}
\usepackage[T1]{fontenc} 	  % Fontencoding

\usepackage{Sweave}


\usepackage[utf8]{inputenc}

% R -----------------------------------------------
\let\proglang=\textsf
\newcommand{\pkg}[1]{{\fontseries{b}\selectfont #1}}

\title{\proglang{R} Basics and Examples - A short introduction}
\author{Markus Baaske}
\institute{Faculty of Mathematics and Computer Science (FSU Jena)}

\begin{document}

\maketitle

\begin{frame}{The R Project for Statistical Computing}
The \proglang{R} project \url{http://www.r-project.org} develops a free
software environment for statistical computing and graphics. \proglang{R}
compiles and runs on a wide variety of UNIX platforms, Windows and MacOS, is mostly used for statistics but can also be
used as a programming (script) language alone. \par
\proglang{R} is organized as a core distribution of base packages which can be
extended by further packages loaded into the a user workspace (or interpreter
global environment). Some useful links are
\begin{itemize}
  \item Tutorials on using \proglang{R} can be
  found at \url{http://www.r-tutor.com/}
  \item Meta search and package
  documentation \url{https://www.rdocumentation.org/}
  \item \proglang{R} CRAN repository for contributed packages:
  \url{https://cran.r-project.org/}
  \item A short reference card \url{https://cran.r-project.org/doc/contrib/Short-refcard.pdf}
\end{itemize}
\end{frame}
%
\begin{frame}[fragile]{\proglang{R} Basics}
\begin{Schunk}
\begin{Sinput}
R> PATH <- getwd()        # get working directory
R> INFO <- Sys.info()     # get system info  
R> objects()              # show all loaded variables
\end{Sinput}
\begin{Soutput}
[1] "INFO" "PATH"
\end{Soutput}
\begin{Sinput}
R> ls()                   # objects in your workspace
\end{Sinput}
\begin{Soutput}
[1] "INFO" "PATH"
\end{Soutput}
\end{Schunk}
Whats is in these objects?
\begin{Schunk}
\begin{Sinput}
R> PATH
\end{Sinput}
\begin{Soutput}
[1] "/home/baaske/workspace/RIntro/doc"
\end{Soutput}
\begin{Sinput}
R> INFO[c("sysname","nodename","user")]
\end{Sinput}
\begin{Soutput}
           sysname           nodename               user 
           "Linux" "baaskelap.rdm.de"           "baaske" 
\end{Soutput}
\end{Schunk}
\textbf{Important:} On quitting, R offers the option of saving the workspace
image, by default in the file "*.RData". Use before ending the R session:
\begin{Schunk}
\begin{Sinput}
R> rm(list=ls())
R> q()
\end{Sinput}
\end{Schunk}
\end{frame}
%

\begin{frame}[fragile]{\proglang{R} Help and vectors}
Getting help:
\begin{Schunk}
\begin{Sinput}
R> help()              # general help
R> ?length             # help for `length`
R> help.search(lapply) # help for function `lapply`
R> help.start()        # start html help system
\end{Sinput}
\end{Schunk}
Vectors:
\begin{Schunk}
\begin{Sinput}
R> 2+2
\end{Sinput}
\begin{Soutput}
[1] 4
\end{Soutput}
\begin{Sinput}
R> round(pi,3)
\end{Sinput}
\begin{Soutput}
[1] 3.142
\end{Soutput}
\begin{Sinput}
R> sqrt(10)
\end{Sinput}
\begin{Soutput}
[1] 3.162
\end{Soutput}
\begin{Sinput}
R> 1000*(1+0.075)^5-1000
\end{Sinput}
\begin{Soutput}
[1] 435.6
\end{Soutput}
\begin{Sinput}
R> sin(c(30,60,90)*pi/180) 
\end{Sinput}
\begin{Soutput}
[1] 0.500 0.866 1.000
\end{Soutput}
\end{Schunk}
\end{frame}
%
\begin{frame}[fragile]{\proglang{R} variables and subsetting}
\begin{Schunk}
\begin{Sinput}
R> a <- 2*3
R> a
\end{Sinput}
\begin{Soutput}
[1] 6
\end{Soutput}
\begin{Sinput}
R> a^2
\end{Sinput}
\begin{Soutput}
[1] 36
\end{Soutput}
\begin{Sinput}
R> b <- a^2
R> a <- c(17,1,3,9)
R> a
\end{Sinput}
\begin{Soutput}
[1] 17  1  3  9
\end{Soutput}
\begin{Sinput}
R> a[2]
\end{Sinput}
\begin{Soutput}
[1] 1
\end{Soutput}
\begin{Sinput}
R> a[c(1,3)]
\end{Sinput}
\begin{Soutput}
[1] 17  3
\end{Soutput}
\begin{Sinput}
R> a[-2]
\end{Sinput}
\begin{Soutput}
[1] 17  3  9
\end{Soutput}
\begin{Sinput}
R> a[2] <- 1
R> a
\end{Sinput}
\begin{Soutput}
[1] 17  1  3  9
\end{Soutput}
\end{Schunk}
\end{frame}
%
\begin{frame}[fragile]{Characters and categories}
\begin{Schunk}
\begin{Sinput}
R> (x <- "Hallo")                     # character vector
\end{Sinput}
\begin{Soutput}
[1] "Hallo"
\end{Soutput}
\begin{Sinput}
R> (y <- factor(c("C","A","C","B")))  # characters as categories
\end{Sinput}
\begin{Soutput}
[1] C A C B
Levels: A B C
\end{Soutput}
\begin{Sinput}
R> (z <- factor(c(1,1,2)))            # numbers as factors
\end{Sinput}
\begin{Soutput}
[1] 1 1 2
Levels: 1 2
\end{Soutput}
\begin{Sinput}
R> (x <- c(1,2,3))                    # distroy x and overwrite
\end{Sinput}
\begin{Soutput}
[1] 1 2 3
\end{Soutput}
\begin{Sinput}
R> x[4]                               # NA = Not Available
\end{Sinput}
\begin{Soutput}
[1] NA
\end{Soutput}
\begin{Sinput}
R> try(x[4])                          # catch error
\end{Sinput}
\begin{Soutput}
[1] NA
\end{Soutput}
\end{Schunk}
\end{frame}

\begin{frame}[fragile]{\proglang{R} object classes}
\begin{Schunk}
\begin{Sinput}
R> class(1.7) # "numeric"
\end{Sinput}
\begin{Soutput}
[1] "numeric"
\end{Soutput}
\begin{Sinput}
R> class(x)   # "character" = character vector
\end{Sinput}
\begin{Soutput}
[1] "numeric"
\end{Soutput}
\begin{Sinput}
R> class(y)   # "factor" categories
\end{Sinput}
\begin{Soutput}
[1] "factor"
\end{Soutput}
\begin{Sinput}
R> class(z)
\end{Sinput}
\begin{Soutput}
[1] "factor"
\end{Soutput}
\begin{Sinput}
R> mode(1.7)
\end{Sinput}
\begin{Soutput}
[1] "numeric"
\end{Soutput}
\begin{Sinput}
R> x <- as.integer(x)
R> class(x)
\end{Sinput}
\begin{Soutput}
[1] "integer"
\end{Soutput}
\begin{Sinput}
R> z <- as.character(z)
R> class(z)
\end{Sinput}
\begin{Soutput}
[1] "character"
\end{Soutput}
\end{Schunk}
\end{frame}
%
\begin{frame}[fragile]{Characters and categories}
\begin{Schunk}
\begin{Sinput}
R> # Save contents of workspace, into the file .RData
R> save.image()
R> # Save into the file archive.RData
R> save.image(file="archive.RData")
R> # save single objects
R> save(x, y,z, file="tmpobj.RData")
R> # save as RDS (could be big data)
R> saveRDS(list(x,y,z),file="myfile.rds")
R> # read as RDS
R> XYZ <- readRDS(file="myfile.rds")
\end{Sinput}
\end{Schunk}
\begin{Schunk}
\begin{Sinput}
R> # attach (reload) to current workspace
R> attach("tmpobj.RData")
R> ls()
\end{Sinput}
\begin{Soutput}
[1] "a"    "b"    "INFO" "PATH" "x"    "y"    "z"   
\end{Soutput}
\end{Schunk}
\end{frame}
%
\begin{frame}[fragile]{\proglang{R} vector repetitions}
\begin{Schunk}
\begin{Sinput}
R> # vectors and repeating components
R> 10:5
\end{Sinput}
\begin{Soutput}
[1] 10  9  8  7  6  5
\end{Soutput}
\begin{Sinput}
R> -1:2
\end{Sinput}
\begin{Soutput}
[1] -1  0  1  2
\end{Soutput}
\begin{Sinput}
R> # a sequence
R> seq(5,10)
\end{Sinput}
\begin{Soutput}
[1]  5  6  7  8  9 10
\end{Soutput}
\begin{Sinput}
R> seq(5,10,by=0.1)
\end{Sinput}
\begin{Soutput}
 [1]  5.0  5.1  5.2  5.3  5.4  5.5  5.6  5.7  5.8  5.9  6.0  6.1  6.2  6.3  6.4
[16]  6.5  6.6  6.7  6.8  6.9  7.0  7.1  7.2  7.3  7.4  7.5  7.6  7.7  7.8  7.9
[31]  8.0  8.1  8.2  8.3  8.4  8.5  8.6  8.7  8.8  8.9  9.0  9.1  9.2  9.3  9.4
[46]  9.5  9.6  9.7  9.8  9.9 10.0
\end{Soutput}
\begin{Sinput}
R> # repeat
R> rep(1:3,times=3)
\end{Sinput}
\begin{Soutput}
[1] 1 2 3 1 2 3 1 2 3
\end{Soutput}
\begin{Sinput}
R> rep(1:3,each=3)
\end{Sinput}
\begin{Soutput}
[1] 1 1 1 2 2 2 3 3 3
\end{Soutput}
\end{Schunk}
\end{frame}
%
\begin{frame}[fragile]{\proglang{R} vector repetitions}
\begin{Schunk}
\begin{Sinput}
R> # replicate
R> x <- 1:5
R> y <- 3:1
R> # multiply elements
R> x*y
\end{Sinput}
\begin{Soutput}
[1]  3  4  3 12 10
\end{Soutput}
\begin{Sinput}
R> (M <- matrix(sample(1:10),nr=5))
\end{Sinput}
\begin{Soutput}
     [,1] [,2]
[1,]    3    6
[2,]    2    9
[3,]    7   10
[4,]    8    4
[5,]    5    1
\end{Soutput}
\begin{Sinput}
R> as.numeric(M%*%c(2,2))
\end{Sinput}
\begin{Soutput}
[1] 18 22 34 24 12
\end{Soutput}
\end{Schunk}
\end{frame}
%
\begin{frame}[fragile]{\proglang{R} data frame object}
\begin{Schunk}
\begin{Sinput}
R> ?data.frame    # help on data frames
R> example(data.frame)   # some examples
\end{Sinput}
\end{Schunk}
Construct a data frame of study courses
\begin{Schunk}
\begin{Sinput}
R> g <- data.frame(StG=c("GTB","MPV","BGM"),Anz=c(75,11,62))
R> g
\end{Sinput}
\begin{Soutput}
  StG Anz
1 GTB  75
2 MPV  11
3 BGM  62
\end{Soutput}
\begin{Sinput}
R> class(g)   # "data.frame" = Datenmatrix
\end{Sinput}
\begin{Soutput}
[1] "data.frame"
\end{Soutput}
\begin{Sinput}
R> names(g)   # categories in g
\end{Sinput}
\begin{Soutput}
[1] "StG" "Anz"
\end{Soutput}
\begin{Sinput}
R> g[,2]	   # 2nd column
\end{Sinput}
\begin{Soutput}
[1] 75 11 62
\end{Soutput}
\begin{Sinput}
R> g[3,]	   # 3rd row
\end{Sinput}
\begin{Soutput}
  StG Anz
3 BGM  62
\end{Soutput}
\begin{Sinput}
R> g[3,2]	   # single element
\end{Sinput}
\begin{Soutput}
[1] 62
\end{Soutput}
\end{Schunk}
\end{frame}
%
\begin{frame}[fragile]{\proglang{R} data frame object}
\begin{Schunk}
\begin{Sinput}
R> g[c(2,3),] # select 2nd and 3rd row
\end{Sinput}
\begin{Soutput}
  StG Anz
2 MPV  11
3 BGM  62
\end{Soutput}
\begin{Sinput}
R> g$Anz      # select category `Anz`
\end{Sinput}
\begin{Soutput}
[1] 75 11 62
\end{Soutput}
\begin{Sinput}
R> g$Anz[1]   # select first element of `Anz`
\end{Sinput}
\begin{Soutput}
[1] 75
\end{Soutput}
\begin{Sinput}
R> # Extending the  data frame
R> (g <- rbind(g,c("GTB",26)))
\end{Sinput}
\begin{Soutput}
  StG Anz
1 GTB  75
2 MPV  11
3 BGM  62
4 GTB  26
\end{Soutput}
\end{Schunk}
\end{frame}
%
\begin{frame}[fragile]{\proglang{R} data frame object}
\begin{Schunk}
\begin{Sinput}
R> # add category
R> (g <- cbind(g,Sem=c(3,1,3,5)))
\end{Sinput}
\begin{Soutput}
  StG Anz Sem
1 GTB  75   3
2 MPV  11   1
3 BGM  62   3
4 GTB  26   5
\end{Soutput}
\begin{Sinput}
R> # add factor level
R> (g <- rbind(g,data.frame(StG=factor("BGOK"),Anz=57,Sem=3)))
\end{Sinput}
\begin{Soutput}
   StG Anz Sem
1  GTB  75   3
2  MPV  11   1
3  BGM  62   3
4  GTB  26   5
5 BGOK  57   3
\end{Soutput}
\end{Schunk}
\end{frame}
%
\begin{frame}[fragile]{\proglang{R} statistics}
\begin{Schunk}
\begin{Sinput}
R> runif(n=10) # uniform on [0,1]
\end{Sinput}
\begin{Soutput}
 [1] 0.20949 0.30704 0.61871 0.63192 0.09593 0.49546 0.38556 0.73465 0.44369
[10] 0.48235
\end{Soutput}
\begin{Sinput}
R> rnorm(n=20,mean=5,sd=2)  # normal distribution
\end{Sinput}
\begin{Soutput}
 [1] 8.487 5.968 4.237 6.551 9.362 4.210 5.725 3.062 9.080 5.094 5.508 3.672
[13] 4.835 1.563 2.030 5.365 4.805 6.113 5.957 7.456
\end{Soutput}
\begin{Sinput}
R> rnorm(n=20)   # standard normal (mean=0, sd=1)       
\end{Sinput}
\begin{Soutput}
 [1] -0.70798  0.33993 -0.57505 -1.00112 -0.08003 -0.17630 -1.13482 -0.48084
 [9] -0.11669  0.09420  0.38246 -1.39561  0.25327 -1.35219  0.95691 -0.22480
[17] -2.95086 -0.22754  1.27911 -0.01134
\end{Soutput}
\begin{Sinput}
R> rpois(n=20,lambda=5) # Poisson with parameter lambda
\end{Sinput}
\begin{Soutput}
 [1] 3 3 4 5 2 7 5 7 4 7 5 6 4 3 6 5 7 8 7 4
\end{Soutput}
\begin{Sinput}
R> rexp(n=20,rate=5) # exponential distribution
\end{Sinput}
\begin{Soutput}
 [1] 0.509424 0.312712 0.349580 0.005942 0.129564 0.026428 0.026867 0.213460
 [9] 0.019561 0.505199 0.153799 0.026285 0.153606 0.300117 0.126952 0.100850
[17] 0.025016 0.104827 0.384372 0.074233
\end{Soutput}
\end{Schunk}
\end{frame}
%
\begin{frame}[fragile]{\proglang{R} statistical characteristics}
\begin{Schunk}
\begin{Sinput}
R> x <- runif(100)
R> min(x)    # Minimum
\end{Sinput}
\begin{Soutput}
[1] 0.01077
\end{Soutput}
\begin{Sinput}
R> max(x)    # Maximum
\end{Sinput}
\begin{Soutput}
[1] 0.9868
\end{Soutput}
\begin{Sinput}
R> mean(x)   # Mean
\end{Sinput}
\begin{Soutput}
[1] 0.5204
\end{Soutput}
\begin{Sinput}
R> median(x) # Median
\end{Sinput}
\begin{Soutput}
[1] 0.5335
\end{Soutput}
\begin{Sinput}
R> var(x)    # variance
\end{Sinput}
\begin{Soutput}
[1] 0.07077
\end{Soutput}
\begin{Sinput}
R> sd(x)	  # standard deviation/error
\end{Sinput}
\begin{Soutput}
[1] 0.266
\end{Soutput}
\begin{Sinput}
R> summary(x) # overview
\end{Sinput}
\begin{Soutput}
   Min. 1st Qu.  Median    Mean 3rd Qu.    Max. 
 0.0108  0.3219  0.5335  0.5204  0.7300  0.9868 
\end{Soutput}
\end{Schunk}
\end{frame}
%
\begin{frame}[fragile]{\proglang{R} loops}
\begin{Schunk}
\begin{Sinput}
R> for(i in 7:9) {print(i)}
\end{Sinput}
\begin{Soutput}
[1] 7
[1] 8
[1] 9
\end{Soutput}
\begin{Sinput}
R> x <- 0
R> while(x < 3){print(x); x <- x+1}
\end{Sinput}
\begin{Soutput}
[1] 0
[1] 1
[1] 2
\end{Soutput}
\begin{Sinput}
R> fun <- function(x) x^2
R> fun(2.3)
\end{Sinput}
\begin{Soutput}
[1] 5.29
\end{Soutput}
\end{Schunk}
Logical ifs
\begin{Schunk}
\begin{Sinput}
R> x <- sample(1:10,size=1)
R> if(x < 5) {print("Yeah!")} else {print("Oh, noo!")}
\end{Sinput}
\begin{Soutput}
[1] "Oh, noo!"
\end{Soutput}
\begin{Sinput}
R> print(x)
\end{Sinput}
\begin{Soutput}
[1] 5
\end{Soutput}
\end{Schunk}
\end{frame}
%
\begin{frame}[fragile]{\proglang{R} Practice}
\begin{enumerate}
 \item Generate standard normal random variables, check characteristics and use
 different sample sizes!
 \item Construct a numeric vector and sort this in ascending order using
 function \textit{sort()}.
 \item Delete first row of data frame `g`. Add some new category. Change number
 of students `Anz` in in `MPV`.
 \item Load data set \textit{airquality}:
\begin{Schunk}
\begin{Sinput}
R> data(airquality)
R> head(airquality)
\end{Sinput}
\end{Schunk}
Check for `NAs` and omit these by functions \textit{is.na()} and
\textit{na.omit()} What is it about? Write a function
 which calculates the means and standard deviations of categories `Ozone`,`Solar.R`, `Wind` and `Temp` from data set
 \textit{airquality} for each month separately. THe function returns a matrix of
 dimensions $5\times 2$ where the first columnstores the means and the second
 the standard errors of all month.
\end{enumerate}
\end{frame}
\end{document}
